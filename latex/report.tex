\documentclass[12pt,a4paper]{article}
\usepackage[UTF8]{ctex}
\usepackage{amsmath}
\usepackage{amssymb}
\usepackage{graphicx}
\usepackage{hyperref}
\usepackage{geometry}
\usepackage{enumitem}
\usepackage{booktabs}
\usepackage{float}

\geometry{left=2.5cm,right=2.5cm,top=3cm,bottom=3cm}

\title{AI社会科学大作业-经济部分}
\author{宋一旸 \quad 许智翔 \quad 刘超然}
\date{2025-12-22}

\begin{document}

\maketitle

\section{背景介绍}

产业集群模拟是一项重要的经济仿真任务,旨在通过计算模型模拟真实世界中企业之间的相互作用、产业链关系和空间分布规律。本研究基于天津产业集群的真实数据,构建了一个可模拟产业集群自然演化过程的计算模型。

\subsection{天津产业集群数据}

本研究使用的数据来源于天津产业集群的真实企业数据。根据\texttt{data}文件夹中的企业分类数据,我们拥有927家企业的详细信息,包括:

\begin{itemize}
    \item \textbf{企业基本信息}:企业名称、成立日期、登记状态等
    \item \textbf{行业分类}:企业所属的7个行业类别(原材料、零部件、电子、电池/电机、整车制造、服务、其他)
    \item \textbf{企业规模}:大型、中型、小型、微型企业的分布情况
    \item \textbf{空间位置}:企业的地理坐标信息,反映产业集群的空间分布特征
    \item \textbf{资本信息}:企业的初始资本规模,总资本约4250万元人民币
\end{itemize}

数据分析显示,天津产业集群呈现出明显的行业分布特征:零部件企业占主导地位(78.5\%),其次是电子企业(13.0\%)和电池/电机企业(5.7\%)。从企业规模来看,微型企业占65.6\%,反映了产业集群以中小企业为主的特征。

\subsection{研究目标}

通过观察数据,我们发现天津产业集群具有以下特征:
\begin{enumerate}
    \item \textbf{行业类别分布}:7个行业类别形成清晰的产业链关系,从上游的原材料到下游的整车制造和服务
    \item \textbf{空间分布特征}:企业在空间上呈现一定的聚集模式,反映了产业集群的地理集聚效应
    \item \textbf{企业生命周期}:企业存在成立、运营、消亡的完整生命周期,约61\%的企业已经停止运营
\end{enumerate}

基于这些观察,我们的研究目标是:
\begin{enumerate}
    \item 构建一个产业链模型,模拟产业集群的自然生长过程(无外界干涉),尽量利用已知的行业分类、空间分布、企业规模等信息
    \item 在此基础上,学习一个强化学习投资策略,探索当有固定投资额度时,如何通过投资或建厂来获得更好的经济优势
\end{enumerate}

\section{方法}

\subsection{环境设定}

天津产业集群面积广大、企业众多,直接模拟整个集群在计算上较为困难。因此,我们抽象概括出了一种更小、更一般、可以用于任意地区产业的交易模型。我们的所有研究都建立在这个模型上。该模型大约比天津集群小20倍,在保持核心经济机制的同时,大大降低了计算复杂度。

\subsubsection{产业链分类和次序}

模型定义了7个行业类别,形成清晰的产业链关系:

\begin{enumerate}
    \item \textbf{原材料(Raw)}:产业链的最上游,为其他行业提供基础原材料
    \item \textbf{零部件(Parts)}:使用原材料生产汽车零部件
    \item \textbf{电子(Electronics)}:使用原材料生产电子控制系统和传感器
    \item \textbf{电池/电机(Battery/Motor)}:使用原材料生产动力系统
    \item \textbf{整车制造(OEM)}:整合零部件、电子和电池/电机,生产最终产品
    \item \textbf{服务(Service)}:为整车制造提供销售、维修等服务
    \item \textbf{其他(Other)}:其他相关行业
\end{enumerate}

产业链的层级关系为:Raw $\rightarrow$ \{Parts, Electronics, Battery/Motor\} $\rightarrow$ OEM $\rightarrow$ Service。每个层级的企业只能从上游层级购买产品,向下游层级销售产品。

\subsubsection{商品和生产流程}

模型采用基于产品的交易系统:
\begin{itemize}
    \item 每个企业根据其行业类别和资本规模,具有不同的生产能力
    \item 生产能力由资本规模和生产比例决定:$生产量 = 资本 \times 生产比例$
    \item 不同行业的生产比例不同:原材料行业为2.5\%,零部件、电子、电池/电机为1\%,整车制造和服务为2.2\%
    \item 企业生产的产品存储在库存中,用于满足下游企业的订单需求
\end{itemize}

\subsubsection{交易流程}

交易流程采用买方驱动模型:
\begin{enumerate}
    \item \textbf{需求计算}:下游企业根据其生产能力和订单需求,计算所需的上游产品数量
    \item \textbf{订单分配}:每个买方企业向最近的K个供应商(默认K=3)发送订单请求
    \item \textbf{供应分配}:供应商根据自身库存和买方距离,按比例分配产品
    \item \textbf{交易执行}:完成交易后,买方支付货款,供应商获得收入
\end{enumerate}

交易量由\texttt{trade\_volume\_fraction}参数控制(默认1\%),表示每个时间步的交易量占企业资本的比例。

\subsubsection{买卖订单的分配}

订单分配遵循以下规则:
\begin{itemize}
    \item 每个买方企业优先选择距离最近的供应商
    \item 考虑运输成本,距离越近的供应商获得越多的订单份额
    \item 供应商的库存和资本限制其能够满足的订单数量
    \item 采用距离加权分配:$分配比例 \propto \frac{1}{距离^2 + \epsilon}$
\end{itemize}

\subsubsection{管理成本}

企业运营需要支付管理成本:
\begin{itemize}
    \item \textbf{运营成本}:与资本规模成正比,$运营成本 = 资本 \times 运营成本率 \times 行业系数$
    \item \textbf{行业系数}:不同行业的运营成本系数不同,原材料行业为1.5,零部件为1.2,电子和电池/电机为1.1,整车制造为0.8,服务为0.3
    \item \textbf{固定成本}:每个时间步支付固定的运营成本(默认-50)
    \item \textbf{资本上限}:企业资本有上限(默认1亿元),超过上限会产生额外的管理成本
\end{itemize}

\subsubsection{固定成本}

除了运营成本外,企业还需要支付:
\begin{itemize}
    \item 每步固定的运营费用(\texttt{fixed\_cost\_per\_step},默认-50)
    \item 资本持有成本:企业持有的资本超过一定比例时,会产生额外的持有成本
    \item 不同行业的资本持有上限不同:原材料、整车制造、服务行业为60\%,零部件、电子、电池/电机为20\%
\end{itemize}

\subsubsection{运输成本的设计}

运输成本采用距离的逆平方定律:
\begin{equation}
运输成本 = \frac{运输成本率 \times 交易量}{距离^2 + \epsilon}
\end{equation}

其中:
\begin{itemize}
    \item 不同行业的运输成本率不同:原材料为0.1,零部件、电子、电池/电机为0.22,整车制造为0.1,服务为0.05
    \item 距离小于\texttt{free\_delivery\_distance}(默认5.0)的交易免运费
    \item $\epsilon$为最小距离常数(默认0.1),防止距离为0时的数值问题
\end{itemize}

这种设计反映了真实世界中运输成本随距离快速增加的特征,鼓励企业在空间上形成产业集群。

\subsubsection{空间设定}

环境设定在二维平面上:
\begin{itemize}
    \item 空间大小:40$\times$40单位(比天津集群小约20倍)
    \item 企业位置:随机分布在空间内,或根据真实数据的地理坐标映射
    \item 最大企业数:360家(可根据配置调整)
    \item 初始企业数:50家(可根据配置调整)
\end{itemize}

\subsection{强化学习设计}

\subsubsection{问题设定}

假设我们有一定金额、固定期数的外来资金,可以用于:
\begin{enumerate}
    \item \textbf{投资现有企业}:向现有企业注入资金,增加其资本规模,提高生产能力
    \item \textbf{建立新企业}:在指定位置和行业类别建立新企业
\end{enumerate}

我们的目标是学习一个投资策略,使得在固定投资预算下,能够获得更好的经济优势。这里的"优势"可以通过以下指标衡量:
\begin{itemize}
    \item 产业集群的总资本增长
    \item 企业数量的增长
    \item 产业链的完整性和稳定性
    \item 空间集聚效应
\end{itemize}

\subsubsection{模型架构}

我们采用基于CNN的特征提取器和MLP策略网络:

\textbf{观察空间处理}:
\begin{itemize}
    \item 将企业位置和资本信息转换为网格地图(Grid Map)
    \item 网格大小:40$\times$40,与空间大小一致
    \item 通道数:7个行业通道 + 1个当前层级通道 = 8通道
    \item 每个网格单元存储该位置所有企业的资本对数和的行业分布
\end{itemize}

\textbf{CNN特征提取器}:
\begin{itemize}
    \item 输入:40$\times$40$\times$8的网格地图
    \item 结构:3层卷积 + 2层全连接
    \item 卷积层:Conv2d(8, 16) $\rightarrow$ Conv2d(16, 32) $\rightarrow$ Conv2d(32, 64)
    \item 池化:每层后接MaxPool2d(2$\times$2)
    \item 全连接:512 $\rightarrow$ 256维特征向量
\end{itemize}

\textbf{策略和价值网络}:
\begin{itemize}
    \item 策略网络(Actor):256 $\rightarrow$ 128 $\rightarrow$ 动作分布
    \item 价值网络(Critic):256 $\rightarrow$ 128 $\rightarrow$ 价值估计
    \item 激活函数:ReLU
\end{itemize}

\subsubsection{观察空间}

观察空间为连续的多维张量:
\begin{itemize}
    \item \textbf{网格地图}:40$\times$40$\times$8的浮点张量
    \item \textbf{行业通道}:7个通道,每个通道存储对应行业在该位置的企业资本对数
    \item \textbf{层级通道}:1个通道,存储当前时间步的产业链层级信息(0-5,归一化到0-1)
    \item \textbf{归一化}:所有值归一化到合理范围,便于神经网络处理
\end{itemize}

\subsubsection{动作空间}

动作空间为连续4维向量:
\begin{equation}
a = [logits_{invest}, logits_{create}, x, y]
\end{equation}

其中:
\begin{itemize}
    \item $logits_{invest}$:投资动作的logit值,用于选择投资还是建厂
    \item $logits_{create}$:建厂动作的logit值
    \item $x, y$:目标位置的坐标(归一化到[-1, 1])
\end{itemize}

动作选择逻辑:
\begin{enumerate}
    \item 根据$logits_{invest}$和$logits_{create}$的softmax概率选择动作类型
    \item 如果选择投资:在$(x, y)$位置附近寻找最近的企业进行投资
    \item 如果选择建厂:在$(x, y)$位置建立新企业,行业类别由当前时间步的层级循环决定(0-5对应6个主要行业)
\end{enumerate}

\subsubsection{算法框架}

采用Proximal Policy Optimization (PPO)算法,使用Stable-Baselines3框架:

\textbf{PPO超参数}:
\begin{itemize}
    \item 学习率:0.0001
    \item 折扣因子($\gamma$):0.99
    \item GAE参数($\lambda$):0.95
    \item 裁剪范围(clip range):0.2
    \item 价值函数系数:0.01
    \item 熵系数:0.05
    \item 最大梯度范数:0.5
\end{itemize}

\textbf{训练设置}:
\begin{itemize}
    \item 并行环境数:32
    \item 每环境收集步数(n\_steps):64
    \item 批次大小(batch\_size):64
    \item 训练轮数(n\_epochs):10
    \item 总训练步数:5,000,000
\end{itemize}

\textbf{环境包装}:
\begin{itemize}
    \item \texttt{EpisodeLengthWrapper}:限制每个episode最大长度为64步
    \item \texttt{MapObservationWrapper}:将企业信息转换为网格地图
    \item \texttt{Monitor}:记录训练过程中的统计信息
\end{itemize}

\subsubsection{奖励设定}

奖励函数设计考虑多个因素:

\textbf{正向奖励}:
\begin{itemize}
    \item \textbf{创建奖励}:成功建立新企业获得奖励(默认0.1)
    \item \textbf{资本增长}:产业集群总资本的增长
    \item \textbf{产业链完整性}:不同行业类别企业的均衡发展
\end{itemize}

\textbf{负向惩罚}:
\begin{itemize}
    \item \textbf{无效坐标惩罚}:动作坐标超出空间范围(-10.0)
    \item \textbf{无效层级惩罚}:选择的行业层级不符合当前时间步(-10.0)
    \item \textbf{无效动作惩罚}:其他无效动作(-10.0)
\end{itemize}

总奖励计算:
\begin{equation}
R_{total} = R_{creation} + \alpha \cdot \Delta Capital + \beta \cdot Penalties
\end{equation}

其中$\alpha$和$\beta$为权重系数,通过实验调整。

\subsubsection{Loss设计}

PPO算法的损失函数包括三部分:

\textbf{策略损失(Policy Loss)}:
\begin{equation}
L^{CLIP}(\theta) = \mathbb{E}_t[\min(r_t(\theta)\hat{A}_t, \text{clip}(r_t(\theta), 1-\epsilon, 1+\epsilon)\hat{A}_t)]
\end{equation}

其中:
\begin{itemize}
    \item $r_t(\theta) = \frac{\pi_\theta(a_t|s_t)}{\pi_{\theta_{old}}(a_t|s_t)}$为重要性采样比率
    \item $\hat{A}_t$为优势函数估计(使用GAE)
    \item $\epsilon=0.2$为裁剪范围
\end{itemize}

\textbf{价值损失(Value Loss)}:
\begin{equation}
L^{VF}(\theta) = \mathbb{E}_t[(V_\theta(s_t) - \hat{V}_t)^2]
\end{equation}

其中$\hat{V}_t$为价值目标(使用n步回报)。

\textbf{熵损失(Entropy Loss)}:
\begin{equation}
L^{ENT}(\theta) = \mathbb{E}_t[H(\pi_\theta(\cdot|s_t))]
\end{equation}

总损失:
\begin{equation}
L(\theta) = -L^{CLIP}(\theta) + c_1 L^{VF}(\theta) - c_2 L^{ENT}(\theta)
\end{equation}

其中$c_1=0.01$(价值函数系数),$c_2=0.05$(熵系数)。

\section{实验}

% 实验部分留空,待补充

\section{结论}

% 结论部分留空,待补充

\end{document}

